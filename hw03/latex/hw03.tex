\documentclass[12pt]{article}

\usepackage{times}
\usepackage{notes}
\usepackage{url}
\usepackage{graphicx}

\setlength{\textwidth}{6.5in}
\setlength{\textheight}{8.9in}
\setlength{\oddsidemargin}{0.0in}
\setlength{\topmargin}{0.05in}
\setlength{\headheight}{-0.05in}
\setlength{\headsep}{0.0in}

\begin{document}

\begin{center}
  {\bf CS 6300} \hfill {\large\bf HW03: Expectimax and Probability} \hfill {\bf Due February 8, 2022}
\end{center}

Please use \LaTeX\ to produce your writeups. See the Homework
Assignments page on the class website for details.

\section{Expectimax}

Bob has found an unfair coin and an unfair 4-sided die. The coin comes
up heads twice as frequently as it comes up tails. The die on the
other hand comes up even twice as often as it comes up odd. I.e. $P(H)
= \frac{2}{3}$, $P(T) = \frac{1}{3}$, $P($x even$) = \frac{1}{3}$, and
$P($x odd$) = \frac{1}{6}$.

After a little thought Bob decides that he can devise a game which he
can never loose in hopes that he can trick his friend Tom out of
soda. Bob tells Tom that Tom can win \$10 by playing. The game
proceeds as follows.

Tom makes the first move. He can either toss the coin or permit Bob to
roll the die. The outcome of the game is -1 if the coin toss results
in heads and 1 if it comes up tails. Otherwise it's the value of the
die, except that it's -2 and -3 for those values. 

In this game Tom wins if the outcome of the game is a positive number
and looses if it's a negative number. \textbf{Note:} The outcome
should never be 0.

\begin{enumerate}

\item Draw the game tree for this game. Don't skip any layers
  (i.e. include chance nodes even where the outcome is guaranteed to
  happen).

\item Determine the value for each node in the tree and give the
    probability for each edge out of a chance node. Show your work.

\item Clearly Bob did not succeed at making a game he could not
    loose. Suggest a non-trivial change to the outcomes that would
    correct this issue and explain why it works. The solution will be
    simple but changing all values to positive numbers would be
    trivial.

\end{enumerate}

\clearpage

\section{Probability}

Sometimes, there is traffic (cars) on the freeway (C=+c).  This could
either be because of a ball game (B=+b) or because of an accident
(A=+a).  Consider the following joint probability table P(A,B,C) for
the domain.

\begin{center}
\begin{tabular}{|c|c|c|c|} \hline
A  & B  & C  & P(A,B,C) \\ \hline
+a & +b & +c & 0.018    \\ \hline
+a & +b & -c & 0.002    \\ \hline
+a & -b & +c & 0.126    \\ \hline
+a & -b & -c & 0.054    \\ \hline
-a & +b & +c & 0.064    \\ \hline
-a & +b & -c & 0.016    \\ \hline
-a & -b & +c & 0.072    \\ \hline
-a & -b & -c & 0.648    \\ \hline
\end{tabular}
\end{center}

  \begin{enumerate}

  \item What is the distribution $P(A,B)$? 

  \item Are A and B independent in this model given no evidence?

  \item What is $P(A|+c)$?

  \item What is $P(A|+b,+c)$?

  \end{enumerate}

\end{document}
